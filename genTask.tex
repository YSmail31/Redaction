\section{Algorithme de génération de tache}
%E. Bini and G. Buttazzo, “Measuring the performance of schedulability tests,” Real-Time Systems, vol. 30, no. 1-2, pp. 129–154, May 2005. 

en 2005, Bini and Buttazzo [ref] ont crée un algorithme appelé UUnifast, il génère des taux d'utilisation de tache pour un ensemble de tache étant donnée un nombre de tache N et un taux d'utilisation total $U_{tot}$.
En utilisant le principe de l'algorithme UUnifast on a abouti à un algorithme pour générer des taches .
\begin{center}
\begin{algorithm}[H]
 \KwData{Taux Utilisation $U_{tot},$ Nombre de Taches N}
 \KwResult{Ensemble de Taches $\Gamma$ \textbf{Ordonnançable}}
 \While{$\Gamma$ non Ordonnaçable}{
  $\Gamma \longleftarrow \emptyset$\;
  $Sum \longleftarrow U_{tot}$\;
  \For{$i = 0$ \KwTo N-1}{
		$NextSum \longleftarrow Sum * Random()^{\dfrac{1}{(n-i)}}$\;
		$UTask \longleftarrow Sum - NextSum $\;
		$T \longleftarrow Random()$\;
		$C \longleftarrow T * UTask $\;
		$D \longleftarrow T * Random(0.75 , 1)$\;
		$tache \longleftarrow CreerTache(C,D,T)$\;
		$\Gamma \longleftarrow \Gamma \cup tache$\;
		$Sum \longleftarrow NextSum$\;
	}
	$T \longleftarrow Random()$\;
	$C \longleftarrow T * Sum $\;
	$D \longleftarrow T * Random(0.75 , 1)$\;
	$tache \longleftarrow CreerTache(C,D,T)$\;
	$\Gamma \longleftarrow \Gamma \cup tache$\;
 }
 \caption{Generation de taches}
\end{algorithm}
\end{center}